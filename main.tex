\usetheme{default}      % save ink
\usefonttheme{default}  % or try serif, structurebold, ...
\useinnertheme{circles} %rectangles, circles, inmargin, rounded

\definecolor{mipBlue}{RGB}{0, 105, 180} % ETH Blue (primary) 
\usecolortheme[named=mipBlue]{structure}

\setbeamertemplate{navigation symbols}{} % Turn of navigation symbols in the bottom right corner

% Footer: It's not nice, but you can add a footer if you want
%\setbeamertemplate{footline}[text line]{%
%	\parbox{\linewidth}{\vspace*{-9pt} \textcolor{mipBlue}{\insertshorttitle \hfill\insertshortauthor\hfill\insertpagenumber}}
%}	

\newcommand{\notesframe}[1]{\note{\textbf{Notes}:\\ #1}}

\newcommand{\newsection}[1]{
	\section{#1}
	% uncomment the following lines to add an extera slide in the begin of every section to show the agenda with the next section title highlighted
%	\begin{frame}%{Agenda}
%		\tableofcontents[currentsection]
%	\end{frame}
} 

\usefonttheme[onlymath]{serif}
\usepackage[english]{babel}
\usepackage[latin1]{inputenc}
\usepackage{eurosym}
\usepackage{amsmath}
\usepackage{amsfonts}
\usepackage{amsbsy}
\usepackage{scrextend}
\usepackage[authoryear,round]{natbib}
\usepackage{lmodern}
\usepackage{accents}
\usepackage{graphicx}
\usepackage{hyperref}
\usepackage{fontawesome}

%\usepackage{fancybox}
%\usepackage{tikz}
%\usepackage{makecell}
%\usepackage{adjustbox}
%\usepackage{tabularx}
%\usepackage{caption}

% using Windows and MikTex, uncomment the following 3 line if you use the package "caption" or "subfig"
%\makeatletter
%\let\@@magyar@captionfix\relax
%\makeatother

% uses symbols instead of numbers for \footnote{text}
\renewcommand{\thefootnote}{\fnsymbol{footnote}}
\newcommand{\mailto}[1]{\href{mailto:#1}{#1}}

\newtheorem{assumption}{Assumption}
\newtheorem{proposition}{Proposition}

\title[Short title]{Long title}
% respect the lexicographic order!
\author[
%short names for the footer
Gersbach, Surname1,	Surname2
]{
	% long name for the title slide
	Hans Gersbach\inst{1}\footnote{\mailto{hgersbach@ethz.ch}} \and
	Name1 Surname1\inst{1} \and
	Name2 Surname2\inst{2}
}
\institute[MIP, other abbreviation]{
	\inst{1} Chair of Macroeconomics: Innovation and Policy at ETH Zurich \and
	\inst{2} Other institution
}
\date{\today}

\begin{document}


	
\begin{frame}
	\thispagestyle{empty}
	\titlepage
\end{frame}
\notesframe{Read the \texttt{beamer} documentation at \url{http://tug.ctan.org/macros/latex/contrib/beamer/doc/beameruserguide.pdf}.}

%\begin{frame}{Agenda}
%	\tableofcontents
%\end{frame}

\newsection{Introduction}
\begin{frame}{Introduction}
	Start with some bullet points:
	\begin{itemize}
		\item First, ...
		\item Second, ...
	\end{itemize}
\end{frame}
\notesframe{
	\begin{itemize}
		\item Note 1
		\item Note 2
	\end{itemize}

}

\newsection{Model}
\begin{frame}{Model}
	Start with some bullet points:
	\begin{itemize}
		\item First, ...
		\item Second, ...
	\end{itemize}
	All important mathematical numbers in one formula
	\begin{eqnarray*}
		\exp\{-i\pi\} + 1 = 0.
	\end{eqnarray*}
	More text, ...
	\begin{definition}
		Content...
	\end{definition}
\end{frame}
\notesframe{
	\begin{itemize}
		\item Note 1
		\item Note 2
	\end{itemize}
	
}

\newsection{Analysis}
\begin{frame}{Analysis}
	Content...
	\begin{theorem}
		Content...
	\end{theorem}
\end{frame}
\notesframe{
	\begin{itemize}
		\item Note 1
		\item Note 2
	\end{itemize}
}

\newsection{Conclusion}
\begin{frame}{Conclusion}
	Content...
\end{frame}
\notesframe{
	\begin{itemize}
		\item Note 1
		\item Note 2
	\end{itemize}	
}
\end{document}